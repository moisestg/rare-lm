\chapter{Rare Word Prediction}

This chapter introduces

\todo{Finish this}

\section{Problem Description}
\label{sec:problemRare}

Talk about the well known problem of static language models and lack of adaptation

\section{The LAMBADA dataset}
\label{sec:lambada}

This dataset was first introduced in \cite{paperno2016lambada} as a challenging test set specifically designed to probe the genuine language understanding of state-of-the-art NLP models, since in words of the authors \textit{``models' effectiveness at picking statistical generalizations from large corpora can lead to the illusion that they are reaching a deeper degree of understanding than they really are''}. And it does so by casting language understanding in
the classic word prediction framework of language modeling. An example of the dataset can be seen in \autoref{fig:lambadaPassage}.

\begin{figure}[H]
	\begin{quote} 		
		 \textbf{Context:} \textit{``Why?'' ``I would have thought you'd find him rather dry,'' she said. ``I don’t know about that,'' said \underline{Gabriel}. ``He was a great craftsman,'' said Heather. ``That he was,'' said Flannery.} \par
		\textbf{Target sentence:} \textit{``And Polish, to boot,'' said $\rule{1.2cm}{0.15mm}$} . \par
		\textbf{Target word:} \textit{Gabriel}
	\end{quote}
	\caption{Example of a LAMBADA passage} \label{fig:lambadaPassage}
\end{figure}

\subsection{Dataset Construction}

LAMBADA was built using the Book Corpus dataset \cite{zhu2015aligning}, which features 5325 unpublished novels and 465 million words. It was divided

\subsection{Analysis}

